\documentclass[
	USenglish,
    ]{scrartcl}

\usepackage{amsmath}
\usepackage{amssymb}
\usepackage{amsthm}
\usepackage{babel}
\usepackage{graphicx}
\usepackage{mathtools}
\usepackage{todonotes}
\usepackage[T1]{fontenc}
\usepackage[utf8]{inputenc}
\usepackage{csquotes}
\usepackage{hyperref}
\usepackage[
	capitalise,
	nameinlink,
	noabbrev,
    ]{cleveref}

% citations
\usepackage[style=alphabetic]{biblatex}
\addbibresource{references.bib}

% better color scheme for hyperref links
\definecolor{linkcolor}{RGB}{51,145,202}
\hypersetup{
	colorlinks=true,
    citecolor=linkcolor,
    linkcolor=linkcolor,
    urlcolor=linkcolor,
}

% mathematical environments
\theoremstyle{plain}
\newtheorem{theorem}{Theorem}
\newtheorem{lemma}[theorem]{Lemma}
\newtheorem{corollary}[theorem]{Corollary}
\newtheorem{observation}[theorem]{Observation}
\newtheorem{claim}[theorem]{Claim}

\theoremstyle{definition}
\newtheorem{definition}{Definition}

\theoremstyle{remark}
\newtheorem*{remark}{Remark}

% title & author
\title{Find a Fancy Title}
% \subtitle{…}
\author{Erika Mustermann}
\subject{\small
    Seminar Thesis for Methods of Algorithm Design, Summer Term 2021\\
    Department of Informatics, Universität Hamburg
}
\date{\today}

\begin{document}
\maketitle
\begin{abstract}
	This is the abstract that should summarize your work in a few
	sentences.
\end{abstract}

\tableofcontents

\section{A First Section}

Think about how to structure your thesis into sections and subsections.

\subsection{Some Examples of Useful Features}

Citing work by \textcite{DBLP:books/lib/Knuth97} as an example.

Also possible without author names~\cite{DBLP:books/lib/Knuth97}.
The bibtex entry in the file \texttt{references.bib} was copied from
\href{https://dblp.org/rec/books/lib/Knuth97.html?view=bibtex}{DBLP}.

\begin{definition}%

	\label{def:example}
	An example definition.
\end{definition}

Example of how to cite \cref{def:example} above.
There are also environments for theorems, lemmas, and such.

Moreover, you can define your own environments.
See the preamble of the \LaTeX{} source for details.

You may use the \verb|\todo| macro to write
\enquote{todos}\todo{Don't forget to remove them before submission!} for
yourself.
\todo[inline]{Todos can also be written inline.}

\subsection{A Second Subsection}
content of subsection a

…

\section{A Second Section}

…

\printbibliography
\end{document}